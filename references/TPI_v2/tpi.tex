%% A sample LaTex template
%% Use and share!

\documentclass[11pt]{article}
\usepackage[breaklinks=true]{hyperref}
\usepackage{url}
\usepackage{graphicx}
\usepackage{amsmath, amsthm, amssymb, amsfonts}
\usepackage{graphicx}
\usepackage{subfigure} % New added
\usepackage{color} % New added
\usepackage{xspace} % New added
\usepackage{authblk}
\usepackage{tikz}
\usetikzlibrary{shapes,arrows}

%Commands to be used regularly % New added
%*******************************************************
\newcommand{\myfloatalign}{\centering} % how all the floats will be aligned
\renewcommand{\bold}[1]{\boldsymbol{#1}} % for boldsymbols
\newcommand{\xt}{(\bold{x},t)}
\newcommand{\lxt}{(x,t)} % xt for x-dimension 
\renewcommand{\sp}[1]{\left({#1}\right)} % for small parentheses 
\let\vaccent=\v % rename builtin command \v{ to \vaccent{
\renewcommand{\v}[1]{\ensuremath{\mathbf{#1}}} % for vectors
\newcommand{\gv}[1]{\ensuremath{\mbox{\boldmath$ #1 $}}}
% for vectors of Greek letters
\newcommand{\uv}[1]{\ensuremath{\mathbf{\hat{#1}}}} % for unit vector
\newcommand{\abs}[1]{\left| #1 \right|} % for absolute value
\newcommand{\avg}[1]{\left< #1 \right>} % for average
\let\underdot=\d % rename builtin command \d{ to \underdot{
\renewcommand{\d}[2]{\frac{d #1}{d #2}} % for derivatives
\newcommand{\dd}[2]{\frac{d^2 #1}{d #2^2}} % for double derivatives
\newcommand{\pd}[2]{\frac{\partial #1}{\partial #2}} % for partial derivatives
\newcommand{\pdd}[2]{\frac{\partial^2 #1}{\partial #2^2}} % for double partial derivatives
\newcommand{\pdc}[3]{\left( \frac{\partial #1}{\partial #2}\right)_{#3}} % for thermodynamic partial derivatives
\newcommand{\ket}[1]{\left| #1 \right>} % for Dirac bras
\newcommand{\bra}[1]{\left< #1 \right|} % for Dirac kets
\newcommand{\braket}[2]{\left< #1 \vphantom{#2} \right|
  \left. #2 \vphantom{#1} \right>} % for Dirac brackets
\newcommand{\matrixel}[3]{\left< #1 \vphantom{#2#3} \right|
     #2 \left| #3 \vphantom{#1#2} \right>} % for Dirac matrix elements
\newcommand{\grad}[1]{\gv{\nabla} #1} % for gradient
\let\divsymb=\div % rename builtin command \div to \divsymb
\renewcommand{\div}[1]{\gv{\nabla} \cdot #1} % for divergence
\newcommand{\curl}[1]{\gv{\nabla} \times #1} % for curl
\providecommand{\norm}[1]{\lVert#1\rVert}
\newcommand{\ebt}[1]{{\color{blue} (Ellad's Comment: #1)}}
\newcommand{\amit}[1]{{\color{red} (Amit's Comment: #1)}}
\newcommand {\eqn}[1] {Eq.~(\ref{#1})}
\newcommand {\eqns}[1] {Eqs.~(\ref{#1})}
\newcommand {\eqnx}[1] {(\ref{#1})}
\newcommand {\Eqn}[1] {Equation~(\ref{#1})}
\newcommand {\sect}[1] {Section~\ref{#1}}
\newcommand {\fig}[1] {Fig.~\ref{#1}}
\newcommand {\Fig}[1] {Figure~\ref{#1}}
\newcommand {\figs}[1] {Figs.~\ref{#1}}
\newcommand {\figx}[1] {\ref{#1}}
\newcommand {\refcite}[1] {Ref.~\onlinecite{#1}}
\newcommand {\refcites}[1] {Refs.~\onlinecite{#1}}
\newcommand {\refcitex}[1] {\onlinecite{#1}}
\newcommand{\tL}{\theta_{\rm L}}
\newcommand{\tR}{\theta_{\rm R}}
\newcommand{\tM}{\theta_{\rm M}}
\newcommand{\sumdd}[3]{\ensuremath{\sum_{\substack{#1 \\ #2}}^{#3}}}
\newcommand{\nbins}{N_{\rm bins}}
\newcommand{\nprofs}{N_{\rm profiles}}
\newcommand{\mprofs}{M_{\rm profiles}}
\newcommand{\aveprofs}{A_{\rm profiles}}
\newcommand{\ndata}{N_{\rm data}}
\newcommand{\deltmd}{\Delta t_{\rm MD}}
\newcommand{\tpi}{TPI\xspace}
\newcommand{\tpicode}{TPI code\xspace}

\newcounter{desccount}
\newcommand{\descitem}[1]{%
    \item[#1] \refstepcounter{desccount}\label{#1}
}
\newcommand{\descref}[1]{\hyperref[#1]{#1}}

\setlength{\parindent}{0pt}
\setlength{\parskip}{2ex plus 0.5ex minus 0.2ex}
\numberwithin{equation}{section}
\hyphenpenalty=5000
\tolerance=1000

\makeatletter
%\doublespacing
%\linenumbers

\title{\Large{\textbf{Thermal Parameter Identification Code\\User Manual}}} 
\author[1]{Amit Singh\thanks{sing0335@umn.edu}}
\author[1]{Ellad~B.~Tadmor\thanks{tadmor@aem.umn.edu}}
\affil[1]{Department of Aerospace Engineering and Mechanics, The University of
Minnesota, Minneapolis, Minnesota 55455}

\begin{document}
\bibliographystyle{hunsrt} % style of bibliography apsrev4-1
%\author{Amit Singh\\Dept of Aerospace Engg \& Mechanics \\
%          University of Minnesota\\\\
%  \\\\sing0335@umn.edu}
\maketitle


%\begin{abstract}
%
%\end{abstract}
\thispagestyle{empty}

\newpage{}
\setcounter{page}{1}
\section{Introduction}\label{sec:Introduction}
This is the User Manual for the MATLAB implementations of the 
\emph{Thermal Parameter Identification} (TPI) method referred to as the ``\tpicode''.
This document is provided as supplemental material to an article on the \tpi method by A.~Singh and E.~B.~Tadmor.
Here, we give a brief overview of the method and describe the \tpicode and its usage.

The \tpi method uses a nonlinear regression analysis to find the thermal
parameters of either a Fourier, Cattaneo-Vernotte (CV), or Jeffreys-type heat
conduction model. The method takes as input the time history of the temperature
of different regions (bins) along the length of a one-dimensional beam. The
Jeffreys-type model also needs the time history of the heat current of all bins as
input. Initially the temperatures of bins are set to a sinusoidal temperature
distribution of the form:
\begin{align}
  \theta_n = \theta_0 + \Delta \theta_0 \cos(2 \pi n/(\nbins-1)), \quad 0 \le n \le
  \nbins-1,
  \label{IC1}
\end{align}
where $\theta_0$ and $\Delta \theta_0$ are the base temperature and amplitude of
the perturbation in the base temperature, respectively. Then the heated nanobeam
is allowed to relax under periodic boundary conditions, $\theta(t)\vert_{n = 0} =
\theta(t)\vert_{n=\nbins-1}, t \ge 0,$ by removing all thermostats. 
The simulations are repeated multiple times with different initial positions and
momenta of the atoms corresponding to the initial condition described in
\eqn{IC1} to obtain multiple realizations (ensembles). Having
established a steady-state sinusoidal temperature distribution, the
the positions and velocities of the atoms are stored `nEnsembles' times at intervals of
$10^4$ MD time-steps to create initial conditions for `nEnsembles' realizations.
Then every realization of the system is allowed to relax
under NVE conditions, and the average bin temperatures and average
$x$-component of heat current vectors are recorded at each time step in files with
the format specified below.
The ensemble files are grouped into blocks (sets) and then a cosine
average of temperature and a sine average of the $x$-component of the heat current vector
along the length of the beam are computed. Then the thermal parameters are
calculated for each block by performing a nonlinear regression analysis over the
average temperature and heat current data. Non-negative thermal parameters obtained
by this approach are retained and averaged to predict the best fit parameters.
The operation of the \tpicode is controlled through a parameter file called `TPI.INPUT'.

The \tpicode performs two processes:
\begin{enumerate}
  \item \textbf{Consolidating temperature profiles}: 
    Temperature profiles in the files provided by the user (in a format described
    below) are stored by averaging over a time interval `deltat' defined in the
    input parameter file. Each file represents a particular realization
    (ensemble). To reduce noise, these files are grouped into blocks, and the
    temperature and heat current data is block-averaged. Then the cosine average of the
    temperature and the sine average of the $x$-component of the heat current vector along the
    length of the beam are computed for the available number of profiles.
  \item \textbf{Parameter extraction process}: Thermal parameters are extracted
    using a nonlinear regression model. Only non-negative parameters are retained.
\end{enumerate}

\section{I/O Files and Parameters}
This section describes the contents and format of the input and output files.

\subsection{Input Files}
The \tpicode requires an input file, ``TPI.INPUT'', which contains the input
parameters that define the program operation. This file also contains the name
of the files, containing temperature and heat current profiles, to which the
regression analysis is applied.

The input file ``TPI.INPUT'' must have the following lines:
\begin{verbatim}
model
Jeffreys
nemdFileCols
3
nEnsembles
50
nblocks
5
nemdFileOption
LAMMPS
nprofs
100
deltat
0.5
theta0
36.2
dtheta0
3.62
nbins
11
binL
15.975
binVol
7247.69775
gamma
6.176189e-06
\end{verbatim}
The file consists of pairs of lines, the \tpicode parameter name followed on the next line
by its value.  The parameters and their possible values are summarized in 
Table~\ref{table:table01} and described below.
\begin{table}
  \caption{\label{table:table01}\tpicode parameters and the corresponding values}
\centering
  \begin{tabular}{c|c}
  \hline
  \hline
    Parameters     &Values\\ \hline
    model          &Jeffreys or CV or Fourier\\
    nemdFileCols   &2 or 3\\
    nEnsembles     &An integer\\
    nblocks        &An integer\\ 
    nemdFileOption &A string value\\
    nprofs         &An integer\\ 
    deltat         &A non-negative real number\\
    theta0         &A non-negative real number\\
    dtheta0        &A non-negative real number\\
    nbins          &An integer\\ 
    binL           &A non-negative real number\\
    binVol         &A non-negative real number\\
    gamma          &A non-negative real number\\
  \hline
  \hline
  \end{tabular}
\end{table}
\begin{description}
  \item[model] The continuum model used for the parameter extraction. 
Three continuum models are supported: the Jeffreys-type, the Cattaneo-Vernotte
(CV), and the Fourier model. The corresponding value for this parameter can
be either `Jeffreys', `CV', or `Fourier'. 
\item[nemdFileCols] The total number of columns in the nonequilibrium Molecular
  Dynamics (NEMD) data files. This parameter 
  must be either 2 or 3. The first, second, and third column correspond to the bin
  number, bin temperature, and $x$-component of the bin heat current vector
  (heat-flux times volume of the bin), respectively. This parameter must be set to 3 for the
  Jeffreys-type model, since both the temperature and heat current data are required
  for this model.
  For either the CV or the Fourier model, its value can be either 2 or 3,
  since these models only require temperature data. The heat current data (if supplied) is
  ignored for these models.
\item[nEnsembles] Total number of realizations (ensembles) available. 
\item[nblocks] Number of blocks (sets) that the ensembles will be grouped into.
  The thermal parameters are extracted for each block and then
  a mean value is taken for all those blocks which successfully achieve
  nonlinear minimization.
\item[nemdFileOption] This parameter specifies the manner in which the NEMD
  files containing the temperature data and possibly heat current data for $\nbins$
  bins are parsed to the code. The parameter can take either `LAMMPS' or
  `MANUAL' as its value. When its value is `LAMMPS' then the names of the
  files must be either temp.\%d.ensemble or tempCurrent.\%d.ensemble, depending
  upon whether the value of nemdFileCols is 2 or 3. Here `\%d' refers to the
  ensemble number, which can vary from $1$ to nEnsembles. Using the LAMMPS
  script, which can be found with the supplementary materials, these files are
  automatically generated when the option `LAMMPS' is used for the NEMD
  simulation. The format for these files is explained below. When the
  value of this parameter is `MANUAL' then the names of all `nEnsembles' files
  should be linewise listed followed by the string MANUAL. An example case
  has been provided with the supplementary materials. It has the `TPI.INPUT'
  files for both options -- `LAMMPS' and `MANUAL'. 
\item[nprofs] The number of NEMD temperature profiles, $\nprofs$, available in
  each NEMD ensemble file.
\item[deltat] Time difference in picoseconds between two consecutive
  temperature profiles recorded in the NEMD ensemble files.
\item[theta0] Base or mean temperature, $\theta_0$, of the initial sinusoidal
  temperature distribution in degrees Kelvin. 
\item[dtheta0] Amplitude, $\Delta\theta_0$, of the sinusoidal temperature
  distribution in degrees Kelvin. A typical value is 10\% of $\theta_0$.
\item[nbins] Total number of bins $\nbins$. 
\item[binL] The length of a bin in \AA.  
\item[binVol] The volume of a bin in \AA$^3$.  
\item[gamma] Volumetric specific heat, $\gamma$, in eV/(\AA$^3$K). 
\end{description}

\subsection{Format for the NEMD data file}\label{sec:format}
An NEMD ensemble file begin with three commented lines serving as the title for
the file. These lines are ignored by the \tpicode for any processing. The file
then contains a series of temperature and possibly heat current profiles
at consecutive time steps separated by blank lines. Each profile provides for all $n$ 
bins along the beam, the bin number, bin temperature (in degrees Kelvin), and possibly
the $x$-component of heat current vector of the bin ($J_x = q_x V$, in eV\AA/ps, where
$q_x$ is the heat-flux and $V$ is the bin volume), in two or three columns.
Thus the file has the following format:
\begin{verbatim}
#<Title 1>
#<Title 2>
#<Title 3>
1  temperature_1  heat_current_1
2  temperature_2  heat_current_2
.  . . . . . . . . . . . . . . .  
.  . . . . . . . . . . . . . . .  
n  temperature_n  heat_current_n
<blank line>
1  temperature_1  heat_current_1
2  temperature_2  heat_current_2
.  . . . . . . . . . . . . . . .  
.  . . . . . . . . . . . . . . .  
n  temperature_n  heat_current_n
<blank line>
etc.
\end{verbatim}
The third column need not be there when either the CV or the Fourier model are
chosen.
\subsection{Output Files}
Following the successful execution of the code, one of the following output files is generated:
``paramJeffreys.txt'', ``paramCV.txt'', or ``paramFourier.txt'', depending
on the thermal model chosen, which contains the extracted thermal parameters.
For the Jeffreys-type model, the file has the following format:
\begin{verbatim}
Printing Jeffreys parameters for each block (set)
========================================
Case  kappa  tau_T  tau_q  residual norm
========================================
..     ..     ..     ..     ..
..     ..     ..     ..     ..
=================================================
Case 1, 2, 3 refers to (lambda_2)^2 > 0, < 0, = 0.
Case 0 means minimum was not found for the block.
=================================================

Printing final Jeffreys-type parameters:

kappa  std_kappa  tau_T  std_tau_T  tau_q  std_tau_q
====================================================
  ..      ..        ..        ..     ..       ..    
====================================================
\end{verbatim}
The first line contains the text ``Printing Jeffreys parameters for each block
(set)''. The second and fourth line are separators. The third line consists of
column headings. The first column is the case obtained after nonlinear
minimization (see Section~III.C of the article for details). The cases 1, 2, 3
refer to $\lambda_2^2 > 0, < 0, = 0$, respectively. Case 0 means that
minimization was not possible for the particular block. This is explained in the
later lines within separators. Thermal parameters are in the second to fourth
column. The last column denotes the residual norm of the nonlinear curve fitting.
At the bottom of the file, the averages of all non-negative parameters with
their standard errors are reported over all blocks for which minimization was
possible. The files ``paramCV.txt'' and ``parametersFourier.txt'' have the same
format, except that for the CV model only the thermal conductivity and one relaxation time
are included, and for the Fourier model only the thermal conductivity is included.

\section{Further discussion}
It is possible that the \tpicode will fail to provide thermal parameters for the
given number of blocks and number of ensemble files. The following suggestions
may help to obtain results:

\begin{itemize}
  \item The values of the parameters `nblocks' and `nEnsembles' can be changed.
  \item The initial guesses for the thermal parameters can be changed. 
    Nonlinear regression can be highly sensitive to the initial
    guesses. A grid of initial guesses is formed for this purpose and nonlinear
    minimization is performed for all grid values. However, it may occur that
    these grid values do not lead to acceptable minimization results. In
    this case, changing the grid values may help.
\end{itemize}
\newpage{}
%\bibitem[Tipler91] {} Tipler, P.A. 1991. \textit{Physics for
%    Scientists and Engineers}, 3rd ed., Worth. Chapter 24.
%\bibitem[ChemGuide]{}
%  http://www.chemguide.co.uk/analysis/masspec/howitworks.html

%\end{thebibliography}

%\bibliography{references}
\end{document}

